\section{Kontrola dostępu}

\begin{frame}{Kontrola dostępu}
		\begin{alertblock}{Kontrola dostępu}
			Kontrola dostępu jest terminem obejmującym \textbf{ogół metod ograniczających dostęp do zasobów}.
			Pomaga to zapobiec nieautoryzowanemu przeglądaniu, modyfikowaniu oraz kopiowaniu zasobów, czyli szeroko pojętej ochronie danych. 
		\end{alertblock}		
\end{frame}


\begin{frame}{Uwierzytelnianie}

\end{frame}

\begin{frame}{2FA}
	
	Tokeny software'owe, np. Google Authenticator. :heart: % warum ich nicht `♥' kann?!?1
	
	Tokeny sprzętowe.
	
	Token ma pewien klucz w pamięci + zna aktualny czas. Jest wtedy w stanie obliczyć tę samą liczbę co serwer. Jeśli się zgadza, to puszcza dalej.
	
	Problem: czasem trzeba synchronizować czas serwera i tokena, jeśli się rozjadą, bo zegar tokena wyprodukowali Chińczycy na eksport.
	
	Albo SMS-ki, ale to głupota.
	
	Por. http://zaufanatrzeciastrona.pl/post/jak-absolutnie-nie-uzywac-tokenow/
	
\end{frame}

\begin{frame}{ACL}
	
	To co w Mumble Server. Nawet niezłe.
	
	Leci od góry do dołu, i jak coś zmatchuje, to wykonuje.
	
\end{frame}

\begin{frame}{Tokeny}
	
	Tu będzie o OAuth2?
	
\end{frame}

\begin{frame}{Niezaprzeczalność}
	
	??? WTF?!
	
\end{frame}

\begin{frame}{Autoryzacja}
	
	Nadawanie konkretnych uprawnień zautentykowanym userom.
	
\end{frame}

\begin{frame}{Cośtam o autoryzowanym dostępie}
	
\end{frame}


