\section{Kontrola dostępu}

\begin{frame}{Kontrola dostępu}
		\begin{alertblock}{Kontrola dostępu}
			Kontrola dostępu jest terminem obejmującym \textbf{ogół metod ograniczających dostęp do zasobów}.
			Pomaga to zapobiec nieautoryzowanemu przeglądaniu, modyfikowaniu oraz kopiowaniu zasobów, czyli szeroko pojętej ochronie danych. 
		\end{alertblock}		
\end{frame}


\begin{frame}{Uwierzytelnianie}
	
	\begin{alertblock}{Uwierzytelnianie}
		Sposób potwierdzenia zadeklarowanej (w procesie \emph{identyfikacji}) przez podmiot tożsamości, np. za pomocą zgodności hasła lub poprawności podpisu pod zaprezentowanym certyfikatem.
	\end{alertblock}
	
	\begin{itemize}
		\item Celem jest uzyskanie określonego poziomu pewności, że podmiot jest rzeczywiście tym z \emph{identyfikacji}.
		\item Niepoprawne tłumaczenia: \emph{autentykacja}, \emph{autentyfikacja}.
	\end{itemize}

\end{frame}

\begin{frame}{2FA}
	
	\begin{alertblock}{2-Factor Authentication}
		Użytkownik oprócz loginu i hasła musi podać jednorazowy kod z fizycznej karty kodów, otrzymany SMS-em albo odczytany z tokena.
	\end{alertblock}
	
	\begin{itemize}
		\item Przy przechwyceniu loginu i hasła, atakujący nie ma i tak dostępu do źródła kodów.
		
		\item Tokeny — zarówno software'owe (np. aplikacja mobilna \href{https://play.google.com/store/apps/details?id=com.google.android.apps.authenticator2}{Google Authenticator}), jak i sprzętowe — mają w pamięci pewien klucz. Na podstawie klucza i aktualnego czasu obliczają tę samą liczbę, co serwer.
	
		\item Problem: synchronizacja czasu serwera i tokena (rzadko).
		
		\item Problem: \href{http://zaufanatrzeciastrona.pl/post/jak-absolutnie-nie-uzywac-tokenow/}{czynnik ludzki}.
	\end{itemize}
	
\end{frame}

\begin{frame}{ACL}
	
	\begin{alertblock}{Access Control List}
		Lista uprawnień dołączona do danego zasobu. Każdy element składa się z przypisania konkretnych uprawnień do uwierzytelnionej grupy/użytkownika.
	\end{alertblock}
	
	Dwa użycia w kontekście bezpieczeństwa komunikacji:
	\begin{enumerate}
		\item dostęp do konkretnych zasobów serwera po uwierzytelnieniu (por. \href{http://wiki.mumble.info/wiki/ACL_and_Groups/English}{Mumble Server}),
		\item w routerach i in. urządzeniach sieciowych do definicji widoczności usług (por. \href{http://www.cisco.com/c/en/us/td/docs/ios/12_2/security/configuration/guide/fsecur_c/scfacls.html}{Cisco}).
	\end{enumerate}
	
\end{frame}

\begin{frame}{OAuth2, Tokeny}
	
	\begin{alertblock}{OAuth 2.0}
		Otwartym standardem autoryzującym umożliwiający użytkownikowi udzielenie zewnętrznej aplikacji (Z) uprawnień dostępu do części zasobów innej aplikacji (A), w której ma on konto.
	\end{alertblock}
	
	Przebieg:
	\begin{enumerate}
		\item (Z) wysyła (A) prośbę o token (z listą żądanych uprawnień),
		\item (A) pokazuje użytkownikowi ekran akceptacji,
		\item użytkownik akceptuje,
		\item (Z) dostaje \emph{tylko} token,
		\item (Z) używa tokenu zwracając się do (A).
	\end{enumerate}
	
	Token pozwala na dostęp tylko do wybranych zasobów i najczęściej wygasa po ustalonym czasie.

\end{frame}

\begin{frame}{Niezaprzeczalność}
	
	\begin{alertblock}{Niezaprzeczalność}
		Brak możliwości wyparcia się swego uczestnictwa w całości lub w części wymiany danych przez jeden z podmiotów uczestniczących w tej wymianie.
	\end{alertblock}
	
\end{frame}

\begin{frame}{Autoryzacja}
	
	\begin{alertblock}{Autoryzacja}
		Nadanie uwierzytelnionym podmiotom konkretnych uprawnień dostępu do danych zasobów.
	\end{alertblock}
	
	Cele:
	\begin{itemize}
		\item kontrola dostępu;
		\item potwierdzenie, że dany podmiot ma uprawnienia do korzytania z danych zasobów.
	\end{itemize}
	
\end{frame}
