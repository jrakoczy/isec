\section{Zapewnienie spójności danych}

\begin{frame}{Spójność danych}
	
	Szalenie ważne jest, abyśmy mogli stwierdzić z dużą dozą pewności, że dane po stronie odbiorcy i nadawcy są dokładnie takie same.
	
	Chcielibyśmy móc to sprawdzić przesyłając znacząco mniej danych niż same dane.
	
	Do tego celu fenomenalnie nadają się kryptograficzne funkcje hashujące.

\end{frame}

\begin{frame}{Kryptograficzne funkcje haszujące}
	
\end{frame}

\begin{frame}{MD5}
	Badziewie.
	
	Da się obejść.
\end{frame}

\begin{frame}{SHA-n}
	
	To można rozbić na poszczególne wersje: SHA-0, SHA-1, itd.
\end{frame}


\begin{frame}{Podpis cyfrowy}
	
\end{frame}

\begin{frame}{Algorytmy podpisów cyforwych}
	
\end{frame}