\section{Certyfikacja}

\begin{frame}{Certyfikacja}
	\begin{alertblock}{Certyfikat klucza publicznego}
		Elektroniczny dokument używany w kryptografii asymetrycznej, będący dowodem własności danego klucza publicznego przez dany rzeczywisty podmiot.
	\end{alertblock}
	\begin{itemize}
		\item Wydawany przez urząd certyfikacji (CA — Certification Authority) po weryfikacji podmiotu.
		\item Użytkownicy korzystający z klucza publicznego opatrzonego certyfikatem mają dużą pewność, że podmiot po drugiej stronie jest rzeczywiście tym, za który się podaje.
	\end{itemize}
\end{frame}

\begin{frame}{Zawartość certyfikatu}
	Certyfikat zawiera m.in.:
	\begin{itemize}
		\item \emph{podpis urzędu certyfikacji},
		\item informacje o podmiocie (np. domena, dla której certyfikat obowiązuje),
		\item przedział czasu ważności,
		\item numer seryjny, użyte algorytmy kryptograficzne, zakres użycia certyfikatu,
		\item certyfikowany klucz publiczny.
	\end{itemize}
\end{frame}

\begin{frame}{Klasyfikacja}
	\begin{alertblock}{Klasyfikacja}
		14:23:32 <@kuba> Klasyfikacja — jakie są typy certyfikatów.
	\end{alertblock}
\end{frame}